\usepackage{fullpage} %
\usepackage[T1]{fontenc} %
\usepackage[latin1]{inputenc} %
\usepackage[obeyspaces]{url} \urlstyle{sf} %
\usepackage[francais]{babel} %
\usepackage{listings}
\usepackage{color}
\usepackage{graphicx}

\lstset{language=C} 
\definecolor{mygreen}{rgb}{0,0.6,0}
\definecolor{mygray}{rgb}{0.5,0.5,0.5}
\definecolor{mymauve}{rgb}{0.58,0,0.82}
\lstset{ %
  backgroundcolor=\color{white},   % choose the background color
  basicstyle=\footnotesize,        % size of fonts used for the code
  breaklines=true,                 % automatic line breaking only at whitespace
  captionpos=b,                    % sets the caption-position to bottom
  commentstyle=\color{mygreen},    % comment style
  escapeinside={\%*}{*)},          % if you want to add LaTeX within your code
  keywordstyle=\color{blue},       % keyword style
  stringstyle=\color{mymauve},     % string literal style
}

%\printanswers % pour imprimer les r�ponses (corrig�)
%\noprintanswers % Pour ne pas imprimer les r�ponses (�nonc�)
\noaddpoints % pour ne pas compter les points
%~ \qformat{\textbf{Question\thequestion}\quad(\thepoints)\hfill} % Pour d�finir le style des questions (facultatif)
\shadedsolutions % d�finit le style des r�ponses
%~ \framedsolutions % d�finit le style des r�ponses
\renewcommand{\solutiontitle}{\noindent\textbf{Solution:}\par\noindent} % D�finit le titre des solutions


